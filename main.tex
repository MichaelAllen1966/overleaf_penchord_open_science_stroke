%%%%%%%%%%%%%%%%%%%%%%%%%%%%%%%%%%%%%%%%%%%%%%%%%%%%%%%%%%%%%%%
% Set up document
%%%%%%%%%%%%%%%%%%%%%%%%%%%%%%%%%%%%%%%%%%%%%%%%%%%%%%%%%%%%%%%

\documentclass[xcolor={usenames,dvipsnames}]{beamer}
\usetheme{Madrid}
\setbeamersize{text margin left=5mm,text margin right=5mm}

% Dark background with non-white words: 
% \usecolortheme{owl}
% \setbeamercolor{normal text}{fg=yellow}
% \setbeamercolor{frametitle}{fg=yellow}
% \usebeamercolor[fg]{normal text}

% Used to create a section slide between section
\AtBeginSection[]{
  \begin{frame}[noframenumbering, plain]
  \vfill
  \centering
  \begin{beamercolorbox}[sep=8pt,center,shadow=true,rounded=true]{title}
    \usebeamerfont{title}\insertsectionhead\par%
  \end{beamercolorbox}
  \vfill
  \end{frame}
}

% Used to create a subsection slide between subsections
% \AtBeginSubsubsection{\frame{\subsubsectionpage}}
\AtBeginSubsection[]{
  \begin{frame}[noframenumbering, plain]
  \vfill
  \centering
  \begin{beamercolorbox}[sep=8pt,center,shadow=true,rounded=true]{title}
    \usebeamerfont{title}\insertsectionhead:\par\phantom{space please}\par\insertsubsectionhead\par%
  \end{beamercolorbox}
  \vfill
  \end{frame}
}


% Remove default navigation symbols and add just  page number
\setbeamertemplate{navigation symbols}{} % Clear default navigation
% \addtobeamertemplate{navigation symbols}{}{%
%     \usebeamerfont{footline}%
%     \usebeamercolor[fg]{footline}%
%     \hspace{1em}%
%     \insertframenumber/\inserttotalframenumber
% }

% Remove from footer the names, institution, date...
% and just leave page number:
\setbeamertemplate{footline}[frame number]


% For manual font size:
\usepackage{anyfontsize}

% For smaller URLs:
\newcommand{\smallurl}[1]{\textcolor{blue}{\fontsize{5pt}{5.8pt}\selectfont \url{#1}}}
%%%%%%%%%%%%%%%%%%%%%%%%%%%%%%%%%%%%%%%%%%%%%%%%%%%%%%%%%%%%%%%




% Title page

\title{Open science in the stroke projects}

\author{Anna Laws, Michael Allen, Kerry Pearn}

%\institute{Overleaf}
\date{November 2022}

\begin{document}

%\frame{\titlepage}

\begin{frame}
\titlepage

\end{frame}


\begin{frame}
\frametitle{Outline}
\tableofcontents
\end{frame}

%%%%%%%%%%%%%%%%%%%%%%%%%%%%%%%%%%%%%%%%%%%%%%%%%%%%%%%%%%%%%%%

\section{Stroke Projects}

\begin{frame}{The emergency stroke pathway}

\begin{center}
\includegraphics[width=0.99\textwidth]{./images/pathway}
\end{center}

\vspace{4mm}
The central problem we are investigating is that there is a NHS target to give clot-busting drugs (\emph{thrombolysis}) to 20\% of patients, but actually only 11\% patients are receiving them (and this varies from 5\% to 25\% between hospitals).

\end{frame}


%%%%%%%%%%%%%%%%%%%%%%%%%%%%%%%%%%%%%%%%%%%%%%%%%%%%%%%%%%%%%%%


\begin{frame}{Stroke projects}

\begin{itemize}
    \setlength\itemsep{2.5mm}
    \item SAMueL: Stroke Audit Machine Learning 
    \begin{itemize}
        \item Emergency stroke clinical pathway simulation
        \item Machine learning to learn and compare clinical decision-making between hospitals
        \item Detailed (disability-level) clinical outcome model
        \item Health economics model
    \end{itemize}
    \item OPTIMIST: OPTimising IMplementation of Ischaemic Stroke Thrombectomy
        \begin{itemize}
            \item Modelling and optimising the pre-hospital emergency stroke pathway.
        \end{itemize}
    \item Mobile Stroke Units (hopefully!)
    \item Geographic modelling (update)
    \item Whole stroke system modelling?
\end{itemize}
    
\end{frame}


%%%%%%%%%%%%%%%%%%%%%%%%%%%%%%%%%%%%%%%%%%%%%%%%%%%%%%%%%%%%%%%
\section{Open Science Tooling in Stroke Work}

\begin{frame}{Caveats}

\begin{itemize}
    \setlength\itemsep{1mm}
    \item You may know all of this!
    \item Learning these tools takes time
    \item We are still learning
    \item We still shout at things at times
    \item Other tools exist
    \item These are probably \emph{not} tools for people who's job keeps them in Microsoft Office most of the day
\end{itemize}

\vspace{2mm}
But.....
\vspace{2mm}

\begin{itemize}
    \setlength\itemsep{1mm}
    \item It works
    \item We just use free/open tools
    \item The benefits, and quality of what may be produced, far outweigh the challenges
\end{itemize}

\end{frame}

%%%%%%%%%%%%%%%%%%%%%%%%%%%%%%%%%%%%%%%%%%%%%%%%%%%%%%%%%%%%%%%

\begin{frame}{Our stack of tools}


\begin{center}
\includegraphics[width=0.75\textwidth]{./images/open_science-2}
\end{center}

\end{frame}

%%%%%%%%%%%%%%%%%%%%%%%%%%%%%%%%%%%%%%%%%%%%%%%%%%%%%%%%%%%%%%%

\begin{frame}{Examples of tools}

\footnotesize

\begin{itemize}
    \item \textcolor{red}{Jupyter Book \& Notebooks:} 
    \smallurl{https://samuel-book.github.io/samuel_shap_paper_1/introduction/intro.html}
    \item \textcolor{red}{GitHub:} 
    \smallurl{https://github.com/samuel-book/samuel_shap_paper_1}
    \item \textcolor{red}{Zenodo:} 
    \smallurl{https://zenodo.org/account/settings/github/}
    \item \textcolor{red}{VS Code}
    \item \textcolor{red}{Overleaf:}
    \smallurl{https://www.overleaf.com/project/636ba04103f728a89c7b5a5d}
    \item \textcolor{red}{HackMD: }
    \smallurl{https://hackmd.io/@N4jCROVmS9SqGmgj3U66XA/HJPwxmtSs}
    \item \textcolor{red}{Zotero: }
    \smallurl{https://www.zotero.org/groups/4707796/mja_stroke/items/H5JR5N99/library}
    \item \textcolor{red}{Open Print Servers:}
    \smallurl{https://www.medrxiv.org/content/10.1101/2020.07.18.20156653v2}
    \item \textcolor{red}{OSF (Open Science Framework):}
    \smallurl{https://osf.io/dashboard}
    \item \textcolor{red}{Streamlit: }
    \smallurl{https://samuel2-stroke-outcome.streamlit.app/Interactive_demo}
    \item \textcolor{red}{BinderHub: }
    \smallurl{https://github.com/MichaelAllen1966/2004_covid_dialysis}
    
    
\end{itemize}


\end{frame}

%%%%%%%%%%%%%%%%%%%%%%%%%%%%%%%%%%%%%%%%%%%%%%%%%%%%%%%%%%%%%%%

\begin{frame}{General lessons}

\begin{itemize}
    \setlength\itemsep{3mm}
    \item Open Science takes time - quality is not necessarily cheap
        \begin{itemize}
            \item We tend to go through, and refine, notebooks several times to check for errors, make them clear, clean, and understandable
        \end{itemize}
    \item Working with others helps
        \begin{itemize}
            \item We pass notebooks around at times - both for coding and summarising
            \item If a mistake gets through, no single person is to blame
            \item Psychological support
        \end{itemize}
    \item Git can be hard! Allow time to get used to it
    \begin{itemize}
            \item Create a branch for each notebook you work on
            \item Commit and push often
            \item Visual Studio Code has nice Git and GitHub integration (sorry Tom)
        \end{itemize}
    \item If you are new to \LaTeX, don't start with presentations (or posters)
\end{itemize}

\end{frame}

%%%%%%%%%%%%%%%%%%%%%%%%%%%%%%%%%%%%%%%%%%%%%%%%%%%%%%%%%%%%%%%
\section{The PenCHORD Way?}

\begin{frame}{The PenCHORD Way?}

\begin{itemize}
    \setlength\itemsep{3mm}
    \item Do we want a set of tools we more actively support and train people in?
    \item \emph{Supportive} not \emph{coercive}
    \item If so, what tools?
    \item What documentation/training/support material?
    \begin{itemize}
        \item Jupyter Book?
        \item \emph{Mattermost} (FOSS Slack alternative0 for shared online 'live' support
        \item Other?
    \end{itemize}
    \item What gaps do we have?
    \item How do we involve everyone in developing the PenCHORD Way?
\end{itemize}

\end{frame}

\end{document}


